% !TEX root = ../../thesis.tex

\chapter{Conservation}
\epigraph{As long as algebra and geometry have been separated, their progress has been slow and their uses limited; but when these two sciences have been united, they have lent each mutual forces and have marched together towards perfection.}{Joseph-Louis Lagrange}

\noindent
In Noether's very influential paper in 1918, named `Invariante Variationsprobleme '~\cite {Noether1918}, she stated one of the most important theorems in conservation. This was to be known as Noether's theorem. She stated, \textit{`Every differentiable symmetry generated by local actions has a corresponding conserved current.'} This innocuous-looking theorem would lead to many years of work to find these conserved currents. The theorem said a little about how to find these quantities for the Euler-Lagrange, but there is an analogous result for Euler-Poincar\'e. These theorems then need to be derived for every new type of Euler-Poincar\'e equation. Noether's theorems apply to the Lagrangian framework. To calculate conserved quantities for the Hamiltonian framework, you go to consider Casimir's and Poisson brackets.\\

\noindent
In this penultimate chapter, we will discuss the conservation properties of the Euler-Poincar\'e and Lie-Poisson equations. We will take two different routes to find the quantities depending on what framework we work under. Firstly, we will return to the Lagrangian framework and study the Noether Theorems for Euler-Poincar\'e theory and then consider a generalisation of these theorems called the Kelvin-Noether theorems which relate to first integrals over a path in our flow. We will then study the Hamiltonian point of view by returning to Lie-Poisson brackets, considering Casimirs and showing that certain functions that make the brackets vanish are conserved. We will also consider the relation between conserved quantities and Reeb graphs, another potential area for exploration in future work. Finally, we will attempt to show numerically that these quantities are conserved and conclude with ideas relating to extending numerical methods with the preceding theory to guarantee that the conserved quantities are conserved.

\section{Noether Theorems}
Noether Theorems can be derived from the variational principle and reduction. The process involves considering the terms we made vanish to produce the equations. We can state and prove the Noether Theorem for Euler-Poincar\'e equations with advected parameters.
\begin{nthm}[Noether Theorem for Euler-Poincar\'e with advected parameters]
  Each symmetric vector field of the Euler-Poincar\'e reduced lagrangian for the infinitesimal variations,
  $$ \d {\vec u} = \dot \nu - \ad_{\vec u} \nu \qquad \d a = -\nu a, $$
  corresponds to an integral of the Euler-Poincar\'e motion and a conserved quantity,
  $$ \dit \int_{M} \ip{\fd \ell {\vec u}}{\nu}\mu = 0. $$
  \label{thm:n_ep_ap}
\end{nthm}
\begin{proof}
  We will consider the derivation of the Euler-Poincar\'e equations for advection parameters. Then jump to the integration by parts of the $\pd u t$ term.
  \begin{align*}
    0 &= \int_{t_1}^{t_2} \int_M \ip{\fd \ell {\vec u}}{\pd \nu t} - \ip{\ad^*_{\vec u} \fd \ell {\vec u}}{\nu} + \ip{a \diamond \fd \ell a}{\nu} \dd t \\
    &= \int_{t_1}^{t_2} \int_M \ip{-\pdt\fd \ell {\vec u} - \ad^*_{\vec u} \fd \ell {\vec u} + a \diamond \fd \ell a}{\nu} \dd t + \int_M\left. \ip{\fd \ell {\vec u}}{\nu}\right|_{t_1}^{t_2} \mu.
  \end{align*}
  We now note that the first pairing is just the Euler-Poincar\'e equations. Hence, we now can write,
  $$ \dit \int_M \ip{\fd \ell u}{\nu} \mu = 0,$$
  as we use the fundamental theorem of calculus of variations.
\end{proof}

\subsection{Vorticity and Helicity}

\noindent
We can now prove two lemmas for the specific case of the Euler Equations in barotropic fluids. The advective quantity is going to be $a = \rho\dd V$ and then the infinitesimal symmetry for $a$ becomes,
$$ \pounds_\eta (\rho \dd V) = \dd (\eta \cont \rho \dd V) = 0. $$
Using rules from the Appendix \ref{app:dg}, this can now be written as for some vector function, $\vec\Psi$,
\begin{equation}
  \eta \cont \rho \dd V = \dd (\vec \Psi \cdot \dd \vec x) = \tcurl \vec\Psi \cdot \dd \vec S.\label{eq:eta_ds}
\end{equation}
\noindent
\begin{nlemma}[Conservation of Vorticity]
  In the Euler-Poincar\'e equations for advected quantities, the following holds and relates to vorticity being conserved,
  \begin{equation}
    \left( \dit + \pounds_{\vec u} \right)\left(\tcurl \frac{1}{\rho}\fd \ell {\vec u} \cdot \dd \vec S\right) = 0.\label{eq:voricity}
  \end{equation}
\end{nlemma}
\begin{proof}
  We start with Noether's theorem for these equations and then perform a calculation.
  \begin{align*}
    0 &= \dit \ip{\fd \ell {\vec u}}{\eta} \\
    &= \dit \int_M \fd \ell {\vec u} \cdot \eta\dd V \\
    &= \dit \int_M \frac{1}{\rho}\fd \ell {\vec u} \cdot \dd \vec x \wedge \eta \cont (\rho \dd V) \\
    &= \int_M \pdt \frac{1}{\rho}\fd \ell {\vec u} \wedge \dd(\vec\Psi \cdot \dd \vec x) + \frac{1}{\rho}\fd \ell {\vec u} \wedge \pdt \dd(\vec\Psi \cdot \dd \vec x) \\
    &= - \int_M \left( \pdt \dd \left( \frac{1}{\rho}\fd \ell {\vec u} \cdot \dd \vec x \right) + \pounds_{\vec u} \dd \left( \frac{1}{\rho}\fd \ell {\vec u} \cdot \dd \vec x \right) \right) \wedge (\vec\Psi \cdot \dd \vec x)  \\
    &= - \int_M \left(\left( \pdt + \pounds_{\vec u} \right) \dd \left( \frac{1}{\rho}\fd \ell {\vec u} \cdot \dd \vec x \right)\right) \wedge (\vec\Psi \cdot \dd \vec x).
  \end{align*}
  Hence, we can say that, as we know, $\vec\Psi \cdot \dd \vec x \ne 0$. Then the result appears after applying \eqref{eq:eta_ds},
  \begin{align*}
    0 &= \left( \pdt + \pounds_{\vec u} \right) \dd \left( \frac{1}{\rho}\fd \ell {\vec u} \cdot \dd \vec x \right) \\
    &= \left( \pdt + \pounds_{\vec u} \right) \left( \tcurl \frac{1}{\rho}\fd \ell {\vec u} \cdot \dd \vec S \right).
  \end{align*}
\end{proof}
\noindent
If we take the example of the variable density Euler equations, we can say that,
$$ \fd \ell {\vec u} = D\vec u^\flat. $$
Then we can rewrite the above as,
$$ \left( \pdt + \pounds_{\vec u} \right) \left( \tcurl \frac{1}{D}\fd \ell {\vec u} \cdot \dd \vec S \right) = \left( \pdt + \pounds_{\vec u} \right) \left( \tcurl {\vec u} \cdot \dd \vec S \right). $$
This makes it clearer that this geometric term is directly referencing vorticity across each. infinitesimal element.\\

\noindent
We now present two significant ideas in this area. We first iterated conserved quantities. Let us consider another vital theorem, called Ertel's theorem.
\begin{nthm}[Ertel's Theorem]
  If a quantity $a$ satisfied the advection equation, and $\eta$ satisfy $\d\eta = \dot \nu + \ad_{\eta}\nu$ for the labelling symmetry. Then $\pounds_\eta a$ is also advected.
\end{nthm}
\begin{proof}
  By a substitution of the advection equation for $a$, we find,
  $$ \pounds_\eta \left(\pdt + \pounds_u \right)a = \left(\pdt + \pounds_u\right)\pounds_\eta a = 0. $$
  This is the advection equation. Hence, $\pounds_\eta a$ is advected by the flow.
\end{proof}
\noindent
Consider a conserved quantity, $c(t)$; the flow carries this. Hence, we can say that $\pounds_u c(t)$ is also conserved by the Euler-Poincar\'e equations. Hence, we can now consider a conserved quantity using the vorticity in \eqref{eq:voricity}. Hence, we can now set the vorticity as our conserved 2-form in \eqref{eq:eta_ds}.
$$ \dd (\vec\Psi \cdot \dd \vec x) = \dd \left( \frac{1}{\rho}\fd \ell {\vec u} \cdot \dd \vec x \right). $$
This will lead us to a new conserved quantity called Helicity. Helicity is a topological invariant sometimes known as the Hopf invariant. It measures the knottedness of the vortex lines and can also be used to calculate the number of linkages. As the vortex lines are baked in the Lagrangian and hence the flow, this quantity is always conserved.
\begin{nlemma}[Conservation of Helicity]
  The Euler-Poincar\'e equations for advected quantities, the following holds and relates to the Helicity being conserved.
  $$ \dit\int_M \frac{1}{\rho} \fd \ell {\vec u} \cdot \tcurl\left( \frac{1}{\rho} \fd \ell {\vec u} \right) \,\dd V. $$
\end{nlemma}
\begin{proof}
  We start at the third step of the previous argument and then move forward differently,
  \begin{align*}
    0 &= \dit \int_M \left(\frac{1}{\rho}\fd \ell {\vec u} \cdot \dd \vec x\right) \wedge \eta \cont (\rho \dd V) \\
    &= \dit \int_M \left(\frac{1}{\rho}\fd \ell {\vec u} \cdot \dd \vec x\right) \wedge \dd\left( \vec\Psi \cdot \dd \vec x \right) \\
    &= \dit \int_M \left(\frac{1}{\rho}\fd \ell {\vec u} \cdot \dd \vec x\right) \wedge \dd \left( \frac{1}{\rho}\fd \ell {\vec u} \cdot \dd \vec x \right) \\
    &= \dit \int_M \frac{1}{\rho}\fd \ell {\vec u} \cdot \tcurl \left( \frac{1}{\rho}\fd \ell {\vec u} \right)\,\dd V.
  \end{align*}
  Therefore, we can write Helicity as,
  $$ \mathcal{H} := \int_M \frac{1}{\rho}\fd \ell {\vec u} \cdot \tcurl \left( \frac{1}{\rho}\fd \ell {\vec u} \right)\,\dd V. $$
\end{proof}
\noindent
Similarly to above consider the Euler equations for variable density, this time where the dimension of the space is greater equal to three. Then we can write Helicity in the more recognisable form,
$$ \dit \int_M \vec u \cdot \tcurl \vec u \dd V. $$


\subsection{Kelvin-Noether Theorem}
We can now find other conservation quantities using similar ideas to the above. Consider the following change of variables,
$$ \oint_{\g_t} \frac{1}{\rho_t}\fd \ell {\vec u} = \oint_{\g_0} \eta^*\left[\frac{1}{\rho_0}\fd \ell {\vec u}\right] = \oint_{\g_0} \frac{1}{\rho_0} \eta^*\left[\fd \ell {\vec u}\right]. $$
Now, we can use the definition of the Lie derivative to write,
$$ \dit (\eta^* \a) = \eta^* \left( \pdt \a + \pounds_{\vec u} \a \right). $$
Hence, we can write,
\begin{align*}
  \dit \oint_{\g_t} \frac{1}{\rho_t}\fd \ell {\vec u} &= \oint_{\g_0} \frac{1}{\rho_0} \eta^* \left( \pdt \a + \pounds_{\vec u} \a \right)\\
  &= \oint_{\g_t} \frac{1}{\rho_t} \left( \pdt \a + \pounds_{\vec u} \a \right)\\
  &= \oint_{\g_t} \frac{1}{\rho_t} \fd \ell a \diamond a.
\end{align*}
\noindent
The last step is from just considering the Euler-Poincar\'e equations for advected parameters. From this, we present the Kelvin-Noether theorem for advected Euler-Poincar\'e,
\begin{nthm}[Kelvin-Noether Theorem for Euler Poincar\'e Equations]
  For the Euler-Poincar\'e equation with advected quantities, we can write the following,
  $$ \dit \oint_{\g_t} \frac{1}{\rho_t}\fd \ell {\vec u} = \oint_{\g_t} \frac{1}{\rho_t} \fd \ell a \diamond a. $$
\end{nthm}
\noindent
We have two remarks,
\begin{remark}
  For basic Euler-Poincar\'e equations, the following is a conserved quantity,
  $$ \oint_{\g} \frac{1}{\rho}\fd \ell {\vec u}. $$
\end{remark}
\begin{remark}
   We can write the Euler-Poincar\'e equations in \textit{Kelvin-Noether form},
   $$ \left( \pdt + \pounds_{\vec u} \right)\fd \ell {\vec u} = \fd \ell a \diamond a. $$
\end{remark}