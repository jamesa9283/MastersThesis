% !TEX root = ../../thesis.tex

\section{Introduction to Geometric Fluid Dynamics}
We will follow Chapter 1 of Arnold and Khesin's `Topological Methods in Hydrodynamics'~\cite{tmih} and, more generally, Holm, Schmah and Stoica's `Geometric Mechanics and Symmetry'~\cite{holm}.\\

\noindent
We will start with some definitions relating to Group Theory and Lie Theory. We can define a group,
\begin{ndefi}[Group on a Manifold]
  \label{def:grp}
  A set $G$ of smooth transformations of a manifold $M$ onto itself is a \textit{group} if,
  \begin{enumerate}
    \item Given two transformations $g, h \in G$, the composition $g \circ h$ is in $G$,
    \item Given some $g \in G$, the inverse, $g^{-1}$, is also in $G$.
  \end{enumerate}
\end{ndefi}

\noindent
We define the group like this because then the definition of the Lie Group becomes obvious. If the functions induced by the two conditions in Definition \ref{def:grp} are smooth, then we have a \textit{Lie group}. There are some pertinent examples of Lie groups that are useful to mechanics. We use the Lie group of length-preserving rotations for rigid body dynamics, $\SO(3)$. For a detailed survey of the Geometric Mechanics of $\SO(3)$, see~\cite{arthur}. However, we are interested in the volume-preserving diffeomorphisms, $\sdm$, for hydrodynamics.
\begin{ndefi}[Diffeomorphism]
  Let $M$ and $N$ be manifolds, then a diffeomorphism is a bijective map $\phi: M \to N$ such that both $\phi$ and $\phi^{-1}$ are differentiable. The group of diffeomorphisms are denoted $\Diff(M)$.
\end{ndefi}
\noindent
We can further extend this to volume-preserving diffeomorphisms,
\begin{ndefi}[Volume Preserving Diffeomorphism]
  Let $\phi : M \to N$ be a diffeomorphism and $\mu$ be the volume element of $M$. Then we say $\phi$ is volume preserving if $\phi(\mu)$ is the volume element of $N$. The group of volume preserving diffeomorphisms are denoted $\sdm$.
\end{ndefi}
\noindent
We say that the volume-preserving diffeomorphism describes fluid. We mean that a diffeomorphism can describe the flow of a particle. Given a terminal point, we can describe all flows of the fluid particle by a diffeomorphism; they show the path the particle takes. We can see this in Figure \ref{fig:diffeos}. In $\SO(3)$, we have smooth left and right actions; this makes $\SO(3)$ a Lie group. However, a diffeomorphism is not smooth by left action. Therefore, they are not a Lie group. They are, however, a topological group or a Fr\'echet group that provides the structure needed to do Geometric Mechanics by right action. This marvellous result means although we don't have a Lie group, we have a Fr\'echet group and can do all of the mathematics required~\cite{tatclg}. \\

\noindent
As an example, we can define the kinetic energy of a particle in a fluid as the following,
$$ E = \frac{1}{2}\int_M \varphi_t^2 \dd x. $$
This turns out to be nicely right invariant. Let $G = \sdm$. If we perform the operation, $R_h : G \to G$, defined by $R_h(g) = gh$, on the kinetic energy, we can write it as the following,
$$ R_hE = \frac{1}{2}\int_M (\phi_t h)^2 \dd x. $$
We note that $h$ is volume preserving and is just a relabelling of the fluid particles, so $R_hE = E$. This is known as the \textit{relabelling symmetry} and is why the rest of this thesis holds. In the Lagrangian framework, we consider blobs of particles, whilst, in the Eularian framework, we consider singular particles. The fact we consider the Lagrangian framework as equivalent to blobs of particles means that if we relabel the particles in the blob, the dynamics don't change. This is the relabelling symmetry. It is equivalent to saying that for $\eta \in \Diff(M)$, $\eta\mu = \mu$ where $\mu$ is the volume form.\\

\noindent
We also note the left multiplication by $h$ can be written as $L_h(g) = hg$.

\begin{figure}[!ht]
\centering
\resizebox{0.6\textwidth}{!}{\input{./img/diffeo.pdf_tex}}
\caption{The path of a fluid particle.}
\label{fig:diffeos}
\end{figure}

\subsection{Adjoint, Coadjoints and Group Actions}
We spend the next subsection generalising and reviewing the content from~\cite{arthur} but for structure-preserving diffeomorphisms. We shall define the inner automorphism that quickly leads to the adjoint representation. We shall also discover the nature of the Lie algebra of $\sdm$.\\

\noindent
We can consider the $L_h$ and the $R_h$ operators we defined before and define the inner automorphism.
\begin{ndefi}[Inner Automorphism]
  The inner automorphism, $A_g : G \to G$ is defined by, $A_h = L_hR_{h^{-1}}$. Given some $g \in G$, $A_hg = hgh^{-1}$ can also be written.
\end{ndefi}
\noindent
This can then be used to talk about the adjoints of the group and algebra. For $\sdm$, our Lie algebra will be the space of divergence-free vector fields in $M$. We also quickly need to note that to differentiate a map on a manifold, we denote it as follows. Given some $F : M \to M$, then the differential at a point $x \in M$ is $\left.F_*\right|_x : T_xM \to T_{F(x)}M$. We can now define our $\Ad_g$ and $\ad_g$ on $\sdm$ and $\mathfrak{g}$.
\begin{ndefi}[Group Adjoint Operator]
  The differential of $A_g$ at the unit is called the group adjoint operator, $\Ad_g$,
  $$ \Ad_g : \mathfrak{g} \to \mathfrak{g} \qquad \Ad_g \xi = (\left. A_{g*} \right|_e)\xi \qquad \qquad \xi \in \mathfrak{g} = T_eG$$
\end{ndefi}
\noindent
We can further define the adjoint orbit,
\begin{ndefi}[Adjoint Orbit]
  Fix $\xi \in \frg$. The set of $\Ad_g\xi$ images of $\xi$ under the action of $\Ad_g$, $g \in G$, is called the adjoint orbit of $\xi$.
\end{ndefi}
\noindent
If $g \in \sdm$, then $\Ad_g\xi = g\xi g^{-1}$ is the structure-preserving diffeomorphisms acting on a vector field. Now if we let $g(0) = e$ and $\dot g(0) = \eta$. Then we can define the adjoint representation of the Lie algebra,
\begin{ndefi}[Adjoint Representation of the Lie algebra]
  If we take the differential of $\Ad$ at the identity, we have the adjoint representation of the Lie algebra,
  $$ \ad : \mathfrak{g} \times \mathfrak{g} \to \mathfrak{g} \qquad \ad_\xi = \ditat 0 \Ad_{g(t)}. $$
\end{ndefi}
\noindent
We can now take the derivative of the group adjoint on $\sdm$. We can then get that $\ad_\xi\eta = -[\xi, \eta]$ for $\xi, \eta \in \mathfrak{X}(M)$ such that $\tdiv \xi = \tdiv \eta = 0$. \\

\noindent
We can further consider the coadjoint versions of the above. These are maps to the cotangent space and the dual to the Lie algebra. The dual to the Lie algebra can be described for arbitrary dimension. Let $G = \sd(M)$ the group of volume-preserving diffeomorphisms on a manifold $M$ with boundary $\partial M$. The commutator of divergence-free vector fields on $M$ is just $-\{v,w\}$. Then, we have the following result.
\begin{nthm}[Dual of $\sv(M)$]
  The Lie algebra, $\frg$, of the group $G$ is naturally identified with the space of closed differential $(n-1)$-forms on $M$ vanishing on $\partial M$. A divergence-free vector field $v$ is identified with the $(n-1)$-form $v \cont \mu$ where $\mu$ is the volume form of $M$. Therefore, the dual to the Lie algebra is $\frg^* = \O^1(M) / \dd \O^0(M)$.
\end{nthm}
\begin{proof}
  The proof here has two parts. Firstly we prove that the space of all $(n-1)$-forms $v \cont \mu$ is naturally identified to $\O^1(M)/ \O^0(M)$. Then we can prove that $\frg \cong \O^1(M)/ \O^0(M)$.\\

  \noindent
  Recall Cartans magic formula, $\pounds_X \a = X \cont \dd \a + \dd (X \cont \a)$. If $\a = \mu$ and $X = {\vec u}$, some vector field, it reduces to just saying, $\pounds_{\vec u}\mu = \dd({\vec u} \cont \mu)$. We are working over the space of volume preserving diffeomorphisms. So, when we consider the Lie derivative of the volume form over an algebra element, we want it to vanish. Hence, we require $v = {\vec u} \cont \mu$ to be closed. Therefore, $v \in \O^1(M) / \O^0(M)$. As $\vec u$ was arbitrary, $v$ is just any $(n-1)$-form, so we have our natural identification.\\


  \noindent
  Now we seek an isomorphism, $\frg \cong \O^1(M)/ \O^0(M)$. This is more complicated and can be found in Theorem 8.3 (pg 42) of `Topological Methods of Hydrodynamics '~\cite {tmih}.
\end{proof}

\noindent
As we are considering fluid flows on these manifolds. If $M$ is simply connected, we can say that the dual algebra, $\frg^*$, is just the vorticities.

\newpage
\subsection{Lie-Poisson Brackets}
As introduced in the introduction, we can write a Hamiltonian system as,
$$ \dot F = \{F, H\}, $$
where $F : M \to \R$ and $H$ is the Hamiltonian. The right-hand side is just the Canonical Lie-Poisson bracket. We can write this as,
$$ \{F, G\} := \sum_i \pd F {q_i} \pd H {p_i} - \pd F {p_i} \pd F {q_i}. $$
This is the simplest Poisson bracket, and so is called the canonical bracket. These brackets are much like the Lie Brackets reviewed in~\cite{holm}. However, it feeds into the theory of conserved quantities.
\begin{ndefi}[Poisson Structure]
  A Poisson structure on a smooth manifold with a bilinear form mapping, $(f,g) \mapsto \{f, g\}$ which satisfies,
  \begin{itemize}
    \item The Jacobi Identity, $\{\{f, g\}, h\} + \{\{g, h\}, f\} + \{\{f, h\}, g\} = 0$,
    \item The Liebniz Identity, $\{f, gh\} = \{f,g\}h + \{f,h\}g$.
  \end{itemize}
\end{ndefi}
\noindent
One of the important notions surrounding these brackets, alongside their stability detection, is Casimirs. We can define a Casimir,
\begin{ndefi}[Casimir]
  Let $M$ be a smooth manifold, then $C : M \to \R$ be a function. We call $C$ a Casimir of this system if,
  $$ \{F, C\} = 0, $$
  for any $F : M \to \R$.
\end{ndefi}


\noindent
There are more complicated brackets for different groups and models. For example, when we consider rigid bodies over $\SO(3)$ we get,
$$ \{f, h\}(\Pi) := -\ip{\Pi}{\left[ \pd f \Pi,\, \pd g \Pi \right]},$$
where $\Pi = R^{-1}\dot R \in \mathfrak{so}(3)$, $R \in \SO(3)$. This is a specific form of Nambu's $\R^3$ Poisson bracket,
$$ \{f, h\} = \nab c \cdot \nab f \ti \nab h. $$
In ideal fluids, the bracket differs. Consider the dual of $\mathrm{SVect(M)}$, then the bracket of this space with respect to some $\a \in \O^1 / \O^0$ is,
$$ \{F, G\}(\o) = -\int_M \a \left( \left[ \fd F \o, \fd G \o\right] \right) \dd V.$$
This is the Arnold bracket. In the section on conserved quantities, we will show that this bracket doesn't conserve enstrophy and provide some alternate brackets.

%TODO Do I want to extend this? I will need more Symplectic Topology to do LP Reduction.
