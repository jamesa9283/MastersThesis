% !TEX root = ../../thesis.tex

\chapter{Conclusion}

This chapter draws the dissertation to a close. In this thesis, we have seen several ideas surrounding the ideas of geometric and topological fluid dynamics and further Geometric Mechanics. We have presented and exposed many results in the area and produced simple examples towards the applications in this area. We have studied boundary conditions on manifolds with boundary and then studied conserved quantities of these systems. We saw examples of axisymmetric flow, the M\"obius strip and rotating stratified fluid. Geometric Mechanics is still a huge area, and my two theses have only touched the surface in this area. My PhD thesis will be based upon this work and bring together many more areas.\\

\noindent
In the final chapter, we saw applications of all the preceding chapters. In Chapter 1, we explored the area's history and presented some relevant pure background of the thesis. We studied a small amount of Symplectic Geometry, an area with vast branches and one I wish we could have explored more. This will be the basis of further work into multisymplectic reduction and the related numerical methods. In the second chapter, we studied reduction theorems on the diffeomorphism group and in both Lagrangian and Hamiltonian frameworks. These presented some extremely nice groundwork for the third chapter and were something that very nicely led from the author's undergraduate thesis to the Ph.D. thesis. This chapter went into great detail to consider the third chapter. In Chapter 3, we studied conserved quantities in three different ways. Firstly, via Noether Theorems, a usual method to derive conserved quantities. The second Casimirs, a slightly classical way to derive the conserved quantities and then a more interesting way via Reeb Graphs. The Reeb graph method provides an interesting approach that hasn't been generalised but provides an idea of how we can find all the Casimirs of more complicated systems.\\

\noindent
Although there are now nearly a hundred pages of Geometric Mechanics written by me, we have again only taken select areas to study. You only have to glance at Darryl Holm's publication list to tell us that. The author undertook some extra work in the area of numerics while writing this thesis; they implemented papers by Cotter and Bridges~\cite{cotter2005general,hamil_pdes} to show that the conserved quantities that were derived produced the expected numerical observations. The methods were based on a finite element approach. The author's PhD thesis will give a broader idea of the area and develop new multisymplectic numerical methods for this problem.\\

% Graphs here??

\noindent
There is also work towards ideas further than Euler's equations. The author spent some time proving theorems and results about a set of equations called the Lagrangian averaged Navier Stokes, which gives a Geometric Mechanics insight into the Navier Stokes problem. These provide interesting theories about different solutions between the Euler Equations and Navier Stokes. This area was discussed in detail with Darryl Holm, and there's a great wealth of knowledge toward ocean modelling using Stochastic Averaging Lagrangian Transport.\\

\noindent
In addition, another whole area towards using variational principles differs from the usual Hamilton's principle. Work by Dr Hamid Alemi Ardakani~\cite{hamid_sloshing,alemi_ardakani_2019} in this area applied mathematics to wave energy generators and uses Geometric Mechanics in general to study waves.\\

\noindent
We conclude this thesis and the author's time at Oxford similarly to my first thesis. Although one year on, a lot more knowledgeable about mathematics and a better mathematician, the author notes that they have equally enjoyed writing this thesis. We have studied some classical results, produced new examples and laid a perfect basis for a Ph.D. This thesis aimed to bring together the skills and knowledge the author attained during their year in Oxford and succeeded. Further, Geometric Mechanics is still a handy tool to the world and one the author will dedicate their life to—solving the more pertinent problems with the most abstract objects. Solving problems need not be easy, but Geometric Mechanics is a toolbox to make it easy.